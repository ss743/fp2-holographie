\section{Summary}

\subsection{Michelson Interferometer}

With the help of the Michelson Interferometer the coherence length of the Laser was measured to be $C_L \geq 36\,\mathrm{cm}$ which was determined to be sufficient for the experiment as this was the maximum path difference feasible on the optical bench. Furthermore, the sensitivity of the interference pattern was identified to be relatively high so that subsequent parts of the experiment were to be performed with extreme diligence.

\subsection{Determination of the Elastic Modulus of the Beams}

The results for the Elastic Modulus of the Beams are presented in the following table:

\begin{table}[h!]
	\centering
	\begin{tabular}{c|c|c|c}
		Material							& Steel (left)	& Brass (middle)	& Aluminum (right)\\ \hline\hline
		measured $E$ [GPa]			& $196\pm5$	& 	$104\pm3$		& $68.5\pm1.6$			\\ \hline
	literature value \cite{staats} $E_{lit}$ [GPa]	& 195			& 100				& 72
	\end{tabular}
	\caption{Results for the Elastic Modulus}
\end{table}

As we can see the measurement of the Elastic Modulus of Steel seems to have been the most exact and corresponds to the literature value within one standard deviation.  The result for brass is still compatible with the literature value as the discrepancy is no bigger than two standard deviations while the value for Aluminum seems to be off by three standard deviations. This can have a number of reasons.

First off, we used a weight with a higher mass to measure the Elastic modulus for steel as we anticipated a higher Elastic modulus for this material. We were able to observe a higher number of minima as for the other two materials, quite possibly as consequence of this. Hence, there was more data available to fit and extrapolate the value from leading to a quite accurate result.

Simultaneously the different reflectivities of the material led to overexposure in the case of Aluminum making the evaluation and determination of the minima more critical. As the errors are derived from the errors on the fit parameters this is not necessarily considered in the specification of the standard deviation. In general, the errors derived from the fit can be questioned regarding their meaningfulness.

Additionally the scaling of the photographs with Gwyddion could only be adjusted to match reality to a certain degree, as an exact method to rescale the images was not available. This would result in a systematic error regarding the position of the minima. As different photographs were used for different materials, varying systematic errors could have been caused.

\subsection{Eigenoscillation of the aluminium plate}

An overview of the resonant frequencies we determined is given in the following table:

{\centering{}
\begin{tabular}{c|c|c}
Mode 		& gemessene Frequenz [Hz] 	& Literaturwert [Hz]\\ \hline\hline
$m=0,\nu=0$	& $441.5\pm0.5$					& 448	\\ \hline
$m=1,\nu=0$	& $1059\pm1.0$				& 983	\\ \hline
$m=2,\nu=0$	& $1716\pm1.0$				& 1592	\\ \hline
$m=2,\nu=1$	& $2061.0\pm1.0$				& 4090	\\ \hline
$m=1,\nu=1$	& $2964\pm5$				& 2854 \\ \hline
$m=3,\nu=1$	& $5383\pm5$				&-
\end{tabular} \\}
\vskip 0.5cm

As discussed previously the expected linear relation to the $x_{m\nu}$ values could be verified. As the calculated frequencies seem to be off by a factor of ten a comparison with these values did not make sense in this context. The comparison with the literature values yields a linear relation between both sets of frequencies with a slope not significantly different from $1$ and an offset not significantly different to $0$ respectively which in a way validates our measurement and suggests that the statistical errors were underestimated. This might be due to the fact that various properties of the set up causing deviations from the theoretical model could not be quantified and taken into account such as the fact that the plate was mounted with eight screws in a holder. This system (or parts of the system) would have been able to oscillate themselves altering the interference pattern observed. Furthermore, the display of the frequency might not have been reliable which is suggested by the fact that the displayed frequency changed over time without altering the set up. Simultaneously the eight screws might not be mounted with the same strength which would result in an asymmetric set up. Additionally, the theoretical model might not strictly apply to reality e.g. due to approximations made.

Another difficulty resulted from the fact, that our hologram was of a lower quality which was concluded to be a result of the bleaching fluid provided. After starting a new container of bleaching chemicals, the holograms could not be bleached successfully always leaving black marks. Due to time constraints and little outlook on getting a better result while still using the same bleacher we relinquished a new recording. Another reason were disturbances by fellow students and accidental movement of a hologram by a third party that prolonged the process. Collectively, this probably led to a mismatch of one of the resonant patterns producing the outlier at $2061\,\mathrm{Hz}$.


\subsection{Fourier Spectroscopy}
In the Fourier part we managed to get some pictures of the cross-correlation of the two slits. The photographs seem to be consistent with the theoretically expected images but since the seen images were very small it's difficult to compare the pictures with the theoretical ones. Additionally it was very difficult to shoot good pictures of the correlation images because the images were very dark caused by the non-complete bleaching and the difficulty to correctly set the camera's focus. Anyway the cross-correlation was much better to see with bare eyes so we could confirm the consistency of our experiment with the theory. As this was a qualitative experiment the angles were given without error.