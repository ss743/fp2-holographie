\section{Summary}

\subsection{Michelson Interferometer}

With the help of the Michelson Interferometer the coherence length of the Laser was measured to be $C_L \geq 36\,\mathrm{cm}$ which was determined to be sufficient for the experiment as this was the maximum path difference feasible on the optical bench. Furthermore, the sensitivity of the interference pattern was identified to be relatively high so that subsequent parts of the experiment were to be performed with extreme diligence.

\subsection{Determination of the Elastic Modulus of the Beams}

The results for the Elastic Modulus of the Beams are presented in the following table:

\begin{table}[h!]
	\centering
	\begin{tabular}{c|c|c|c}
		Material							& Steel (left)	& Brass (middle)	& Aluminum (right)\\ \hline\hline
		measured $E$ [GPa]			& $196\pm5$	& 	$104\pm3$		& $68.5\pm1.6$			\\ \hline
	literature value \cite{staats} $E_{lit}$ [GPa]	& 195			& 100				& 72
	\end{tabular}
	\caption{Results for the Elastic Modulus}
\end{table}

As we can see the measurement of the Elastic Modulus of Steel seems to have been the most exact and corresponds to the literature value within one standard deviation.  The result for brass is still compatible with the literature value as the discrepancy is no bigger than two standard deviations while the value for Aluminum seems to be off by three standard deviations. This can have a number of reasons.

First off, we used a weight with a higher mass to measure the Elastic modulus for steel as we anticipated a higher Elastic modulus for this material. We were able to observe a higher number of minima as for the other two materials, quite possibly as consequence of this. Hence, there was more data available to fit and extrapolate the value from leading to a quite accurate result.
Simultaneously the different reflectivities of the material led to overexposure in the case of Aluminum making the evaluation and determination of the minima more critical. As the errors are derived from the errors on the fit parameters this is not necessarily considered in the specification of the standard deviation. 