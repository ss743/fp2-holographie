\section{Introduction and Objective}

Holography is a procedure used to obtain three dimensional images of objects. Unlike photographies that only save information about the intensity of the light and therefore only produce a two dimensional image, holograms also store phase information using an interference based technique. 
Since Denis Gabor developed the theoretical fundament for the holography technique in 1947 and the development of the Laser providing a coherent light source in 1963 and therefore making holography a feasible method, there have been multiple applications making great use of holograms.
A few examples are holograms on paper money to make it more difficult to copy it or the use of holography in archeology to protect finds while still being able to analyze them or, as in case of this experiment, very precise measurements.\\
This experiment is split up into four parts. In a first part the coherence length of a Helium Neon Laser used as light source in all further parts of the experiment is determined using a Michelson Interferometer. In the second part we measure the elastic modulus of three beams made of aluminum, brass and steel using a double exposure holography technique. The objective of the third part of the experiment is to determine the resonant frequencies of an Aluminum plate which is done by using time average holography. In the last part we use Fourier interferometry, a technique based on basic Fourier Optics, to determine the Cross-Correlation of a slit with a slit tilted to the first one. We will use rotations from $0-90\degree$ in $15\degree
$ steps.\\
In order to conduct and understand this experiment a solid background in wave optics, Fourier optics, different Holography techniques as well as understanding of the working of the experimental apparatus is required and will be provided in the following.