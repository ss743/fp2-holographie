\section*{Abstract}

Holography is a technique making the recording of 3D images feasible by storing phase information in addition to the usually stored amplitude information of the Electric field using a reference beam. We conduct four different measurements: a) Studying the Coherence properties of the light source, b) Determining the Elastic Modulus of three different materials namely brass, steel and aluminium, c) Identifying the resonant frequencies of an Aluminium plate and d) Observing the Cross-Correlation of two Slits using Fourier Spectroscopy. The coherence properties were determined to be sufficient and the following results were obtained for the subsequent quantitative parts of the experiment:

\begin{table}[h!]
	\centering
	\begin{tabular}{c|c|c|c}
		Material							& Steel (left)	& Brass (middle)	& Aluminum (right)\\ \hline\hline
		measured $E$ [GPa]			& $196\pm5$	& 	$104\pm3$		& $68.5\pm1.6$			\\ \hline
		literature value \cite{staats} $E_{lit}$ [GPa]	& 195			& 100				& 72
	\end{tabular}
	\caption{Results for the Elastic Modulus}
\end{table}

\begin{table}
	\centering
	\begin{tabular}{c|c|c}
		Mode 		& Measured frequency [Hz] 	& Literature value [Hz]\\ \hline\hline
		$m=0,\nu=0$	& $441.5\pm0.5$					& 448	\\ \hline
		$m=1,\nu=0$	& $1059\pm1.0$				& 983	\\ \hline
		$m=2,\nu=0$	& $1716\pm1.0$				& 1592	\\ \hline
		$m=2,\nu=1$	& $2061.0\pm1.0$				& 4090	\\ \hline
		$m=1,\nu=1$	& $2964\pm5$				& 2854 \\ \hline
		$m=3,\nu=1$	& $5383\pm5$				&-
	\end{tabular}
	\caption{Results for the resonant frequencies}
\end{table}

In the fourth part we could confirm the consistency of our experiment with the theory discussed in the Theoretical Background section of the experiment.