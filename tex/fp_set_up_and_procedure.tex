\section{Experimental Set-up and Procedure}

\subsection{Part 1: Michelson interferometer}\label{sec:michelson}

As we have seen interference is crucial for this experiment (see \ref{Interference} ) and as we have further discussed coherence is an essential condition for that (see \ref{Coherence section}). So in the first part of the experiment we determine the Coherence Length of the light source to make sure that the path difference for object and reference beam won't be bigger than that in all following parts. Furthermore we will test how sensitive the set up is to disturbances.\\
As previously mentioned we will be using a Michelson interferometer for this part (see figure \ref{fig: Michelson}). The light beam emitted from the Helium-Neon Laser encounters a beam splitter resulting in two beams propagating independently from one another. They are then reflected back on the beam splitter with a mirror, with both mirrors adjustable in distance (in a classical Michelson set up it is often just one mirror that is adjustable while the other is placed at a fixed distance from the beam splitter). After passing through the beam splitter they then hit a screen where an interference pattern is observed due to the path difference of both beams. However, the interference pattern will only be observed as long the beams are coherent which offers a straight forward way to measure the Coherence length: During the experiment we will change the path difference by moving the mirrors and thereby determine the Coherence length. Furthermore we will study the influence of disturbances such as noise, movement and a fire lighter.


\graTwo[0.5]{Coherence_length}{aufbau_michelson}{Set-up of Meassuring the Coherence Length with a Michelson interferometer \label{fig: Michelson} }


\subsection{Part 2: Double Exposure Hologram - Elastic Modulus of a Beam \label{DEH}}

The set-up used for this part of the experiment can be seen in figure \ref{DEH!}. The alignment of the spatial filters is the crucial part of the set-up and can take some time. A technique to get the spatial filters aligned is given in \cite{anleitung}.
The ratio of the intensities of reference and object beam could be adjusted by turning the beam splitter disk and should ideally be 10:1 at the position of recording the hologram. This can be verified with a photo diode. However, the photo diode turned out to be highly unreliable so that the ratio was estimated with the eye. A guideline can be that the reflection of the beam should just be bright enough to still be distinguishable from the reference beam on the photo plate which should be placed at Brewster angle with respect to the reference beam to minimize reflection. For glass this angle is $56 \degree$. Both beams should hit the centre of the photo plate.  

\graXTwo[0.5]{Versuchsaufbau_2}{aufbau-2}{Set-up for the double-exposure hologram}{top: Set-up for taking the holograms in part 2 and 3 of the experiment bottom: Set-up for taking the double-exposure hologram \label{DEH!}}

For the Double Exposure hologram three baths with developer, bleacher and water are prepared. Based on experience, we chose the following time intervals for the process of recording the hologram:

\begin{itemize}
	\item Exposure: $2 \times 5\,\mathrm{minutes}$
	\item Development: $3\,\mathrm{minutes}$
	\item Pre-watering: $10\,\mathrm{seconds}$
	\item Watering: $2\,\mathrm{minutes}$
	\item Bleaching: $15\,\mathrm{minutes}$
	\item Pre-watering: $10\,\mathrm{seconds}$
	\item Watering: $10\,\mathrm{minutes}$
	\item Cleaning: $1\,\mathrm{minutes}$
\end{itemize}

Once the process was successful and a hologram could be observed, photographs were taken with a digital camera of the hologram and of the beams with a ruler for scaling from the same perspective to ensure the same scaling for all images.



\subsection{Part 3: Real-time Holography - Oscillations of an Aluminium Plate \label{RTH}}

The set up used for this part of the experiment was nearly identical as in the previous section and can be seen in figure \ref{setup3}. Hoewever, here we use a glass plate in a flooding system as recording medium and the object is a round aluminum plate. 
The glass plates were left in a water bath over night before performing the experiment. The set-up was slightly adjusted as the aluminum plate has a different reflectivity and hence the ratio  between object and reference beam needs to be realigned. In order to record the hologram the times for the different parts of the process were slightly varied:

\begin{itemize}
	\item Soaking:  $10\,\mathrm{minutes}$
	\item Exposure: $5\,\mathrm{minutes}$
	\item Development: $3\,\mathrm{minutes}$
	\item Pre-watering: $10\,\mathrm{seconds}$
	\item Watering: $2\,\mathrm{minutes}$
	\item Bleaching: $45\,\mathrm{minutes}$
	\item Pre-watering: $3 \times 30\,\mathrm{seconds}$
	\item Watering: $1\,\mathrm{minutes}$
\end{itemize}

\graX[0.7]{aufbau3}{Experimental Set-up part 3}{Set-up for taking the real-time hologram }{{Set-up for taking the real-time hologram \label{setup3}}

To observe the hologram, the flooding system was refilled with water. The set-up in this part is extremely sensitve to disturbance so the hologram needs to be recorded and observed with extreme caution. Once a hologram was recorded successfully a speaker was placed behind the aluminum plate which was illuminated with the laser in pulsed mode (which can be achieved with the Pockels Cell) and connected to an oscilloscope to observe and optimize the settings. The necessary wiring can be seen in figure \ref{wiring} as well as an ideal signal. 

It is crucial to choose a relatively short pulse duration to observe the interference patterns. Furthermore the settings can be optimized by fine tuning the frequency as well as varying the delay and volume of the speaker.

\graX[0.4]{aufbauschaltung}{ Wiring for the resonance frequencies}{ Wiring for the resonance frequencies \label{wiring}}
\subsection{Part 4: Fourier Interferometry \label{FI}}

The set up used in this part can be seen in figure \ref{setup4}. In this part of the experiment we make particular use of the fact that the Fourier Transform of an object can be observed in the back focal plane of a length as well as the convolution theorem (see \ref{Convo1}). We altered the previous set-up and placed lenses behind the spatial filters to achieve a parallel beam. The reference beam then hits the glass plate without encountering further optical components while a rotatable slit is placed in the object beam. After passing through the slit, the beam then passes through a lens placed in a distance corresponding to the focal length of both slit and glass plate so that the Fourier Transform is recorded. The intenstiy density of both beams should be adjusted to be about the same. Again, the recording times were adjusted and we used a dry plate this time.:

\begin{itemize}
	\item Exposure: $8\,\mathrm{minutes}$
	\item Development: $3\,\mathrm{minutes}$
	\item Pre-watering: $10\,\mathrm{seconds}$
	\item Watering: $2\,\mathrm{minutes}$
	\item Bleaching: $45\,\mathrm{minutes}$
	\item Pre-watering: $3 \times 30\,\mathrm{seconds}$
	\item Watering: $1\,\mathrm{minutes}$
\end{itemize}

\graTwo[0.6]{Versuchsaufbau_4}{aufbau4}{Experimental set-up for the cross-correlation measurement \label{setup4}}

A fourth lens was then placed in the distance of its focal length behind the plate to observe the hologram. Cross-Correlation was recorded in $15\degree$ rotation steps of the slit with a digital camera. However, the hologram was extremely small and difficult to observe. It is important to have sufficient space behind the recording apparatus to place several lenses to magnify the image further if necessary.