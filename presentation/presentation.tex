\documentclass{beamer}

\usepackage[utf8]{inputenc}
\usepackage[T1]{fontenc}
\usepackage{amsmath,amssymb,amstext}
\usepackage{mathrsfs}
\usepackage{siunitx}
\usepackage{multimedia}
\usepackage{nccfoots}

\usepackage{array}
\newcolumntype{P}[1]{>{\centering}p{#1}}

\usepackage{overpic}

\usepackage{ufcd}
\usepackage[english]{babel}
\usepackage[backend=biber,backref=false,style=numeric-comp, sorting=none,block=ragged,firstinits=true]{biblatex}

\addbibresource{fp_refs.bib}

\usetheme{ufcd}

\newcommand{\gra}[3][]{
	\begin{table}
	\centering
	\begin{tabular}[width=\textwidth]{c}
		\includegraphics[width=#1\textwidth]{../figures/#2.png}\\
		\small #3
	\end{tabular}
	\end{table}
}
\newcommand{\graTwo}[5][]{
	\begin{table}
		\centering
		\begin{tabular}[width=\textwidth]{cc}
			\includegraphics[width=#1\textwidth]{../figures/#2.png}&
			\includegraphics[width=#1\textwidth]{../figures/#3.png}\\
			\small #4&\small #5
		\end{tabular}
	\end{table}
}
	
\newcommand{\graFour}[9][]{
	\begin{tabular}[width=\textwidth]{cc}
		\includegraphics[width=#1\textwidth]{../figures/#2.png}&
		\includegraphics[width=#1\textwidth]{../figures/#3.png}\\
		\small #6&\small #7\\
		\includegraphics[width=#1\textwidth]{../figures/#4.png}&
		\includegraphics[width=#1\textwidth]{../figures/#5.png}\\
		\small #8&\small #9
	\end{tabular}
	}
\newcommand{\graTwoOne}[4][]{
	\begin{table}
		\centering
	\begin{tabular}[width=\textwidth]{cc}
		\includegraphics[width=#1\textwidth]{../figures/#2.png}&
		\includegraphics[width=#1\textwidth]{../figures/#3.png}
	\end{tabular}
	{#4}
	\end{table}
}
\newcommand{\nlOne}{\\&\small }
\newcommand{\nlTwo}{\\&\small }
\newcommand{\degree}{^\circ}

\setbeamerfont{footnote}{size=\tiny}


\title{Holography}
\author{Saskia Bondza \& Simon Stephan}
\date{04.04.2017}

\begin{document}
\maketitle
\frame{\tableofcontents}
\section{Theoretical Background}
\frame{\tableofcontents[currentsection]}
\begin{frame}
	\frametitle{Holography - What is it?}
	\gra[0.7]{holo-schach}{}
	\Footnotetext{}{http://www.holoworld.com/holo/images/hologram.jpg}
\end{frame}
\begin{frame}
	\frametitle{Holography - What is it?}
	\begin{tabular}{p{5cm}|p{5cm}}
		\textbf{Photography}&\textbf{Holography}\\\hline
		2-dimensional images&3-dimensional images\\\hline
		stores only amplitude information&stores amplitude and phase information\\\hline
		can be studied at any light&can be studied only at special conditions
	\end{tabular}
\end{frame}

\begin{frame}
	\frametitle{Interference}
\end{frame}
\begin{frame}
	\frametitle{Coherence}
\end{frame}
\begin{frame}
	\frametitle{Hologram of an object point}
\end{frame}
\begin{frame}
	\frametitle{Types of Holography}
\end{frame}
\begin{frame}
	\frametitle{Reconstruction of a Hologram}
\end{frame}
\begin{frame}
	\frametitle{Holographic Interferometry}
\end{frame}
\begin{frame}
	\frametitle{Apparatus}
\end{frame}
\section{Experiments}
\frame{\tableofcontents[currentsection]}
\subsection{Michelson Interferometer}
\frame{\tableofcontents[currentsubsection]}
\begin{frame}
	\frametitle{Set Up}
	\gra[0.85]{Coherence_length}{}%Set Up of the Michelson Interferometer}
\end{frame}
\begin{frame}
	\frametitle{Set Up}
	\begin{figure}
		\centering
		\begin{overpic}[width=0.85\textwidth,tics=20]
			{../figures/aufbau_michelson.png}
			\put(20,37){\footnotesize\textcolor{white}{Mirror 1}}
			\put(30,17){\footnotesize\textcolor{white}{Mirror 2}}
			\put(40,36){\footnotesize\textcolor{white}{Beam Splitter}}
			\put(67,37){\footnotesize\textcolor{white}{Laser}}
		\end{overpic}
	\end{figure}
\end{frame}
\begin{frame}
	\frametitle{Interference}
	\gra[0.67]{michelson1}{Interference in the Michelson interferometer}
	\centering\movie[externalviewer]{\includegraphics[width=0.6cm]{../figures/play.png}Movie of the disturbance by jumping}{../figures/michelson.mov}
\end{frame}
\subsection{Double Exposure Hologram}
\frame{\tableofcontents[currentsubsection]}
\begin{frame}
	\frametitle{Set Up}
	\gra[0.85]{Versuchsaufbau_2}{}%Set Up of the Michelson Interferometer}
\end{frame}
\begin{frame}
	\frametitle{Set Up}
	\begin{figure}
		\centering
		\begin{overpic}[width=0.85\textwidth,tics=20]
			{../figures/aufbau-2.png}
			\put(10,10){\footnotesize\textcolor{white}{Mirror 1}}
			\put(22,20){\footnotesize\textcolor{white}{Spatial Filter 1}}
			\put(60,57){\footnotesize\textcolor{white}{Mirror 2}}
			\put(60,37){\footnotesize\textcolor{white}{Spatial Filter 2}}
			\put(50,10){\footnotesize\textcolor{white}{Beam Splitter}}
			\put(88,13){\footnotesize\textcolor{white}{Laser}}
			\put(26,45){\footnotesize\textcolor{white}{Object}}
			\put(10,35){\footnotesize\textcolor{white}{Photo Plate}}
		\end{overpic}
	\end{figure}
\end{frame}
\begin{frame}
	\frametitle{Double Exposure Hologram}
	\gra[0.8]{staebe4}{Interference pattern on the beams}
\end{frame}
\begin{frame}
	\frametitle{Calculation of the Elastic Modulus}
	Blablabla
\end{frame}
\subsection{Real Time Hologram}
\frame{\tableofcontents[currentsubsection]}
\begin{frame}
	\frametitle{Real Time Hologram}
	Blablabla
\end{frame}
\begin{frame}
	\frametitle{Oscillations of an Aluminium Plate}
	Blablabla
\end{frame}
\subsection{Fourier Interferometry}
\frame{\tableofcontents[currentsubsection]}
\begin{frame}
	\frametitle{Fourier Optics - Huygen's Principle and Fraunhofer Diffraction}
	\gra{Huygen}{Propagation of Waves based on Huygen's Principle \footfullcite{http://web.mit.edu/viz/EM/visualizations/coursenotes/modules/guide14.pdf}}
\end{frame}
\begin{frame}
	\frametitle{Fourier Optics - Huygen's Principle and Fraunhofer Diffraction}
	
	Fresnel-Kirchhoff-Integral Formula:
	\begin{align}
	U(x_0) &= \frac{1}{\lambda L} C \mathscr{F}{g(x, y)}    = \frac{1}{\lambda L} C   \int\limits_{-\infty}^{\infty}  g(x,y)e^{-ikx}dx    &  C  \cdot C^* &= 1                         \label{FKK}
	\end{align}
	
	Intensity Distribution in the Far field:
	\begin{align}
	I=|U(x_0)|^2=\left| \int\limits_{-\infty}^{\infty} g(x,y)e^{-ikx}dx \right|^2
	\end{align}
\end{frame}
\begin{frame}

\end{frame}
\begin{frame}
	\frametitle{Convolution Theorem}
	\textbf{Convolution} is a mathematical operation computing a third function out of two functions $f$ and $g$ that can be imagined as ``smearing'' function $f$ with $g$:
	\begin{align}
	(f*g)(t)&=\int\limits_{-\infty}^{\infty} f(\tau)\cdot g(t-\tau) d\tau
	\end{align}
	Using the definition of the Fourier transfrom, one can easily prove:
	\begin{align}
	\mathscr{F}(f*g)&=const. \mathscr{F}(f)\cdot \mathscr{F}(g)
	\end{align}
\end{frame}
\begin{frame}
	\frametitle{Convolution Theorem}
	\gra[0.7]{Convolution}{Illustration of the Convolution theorem with the example of the double slit}
\end{frame}
\begin{frame}
	\frametitle{Cross- and Autocorrelation}
\textbf{Cross-Correlation} describes the similarity of two functions dependent on the displacement of one relative to another:
	\begin{align}
	(f*g)(t)=\int\limits_{-\infty}^{\infty} f(\tau)\cdot g(t+\tau) d\tau \label{corr}
	\end{align}
\textbf{Auto-Correlation} describes the self-similarity of a function with a displaced version of itself
\end{frame}
\begin{frame}
	\frametitle{Cross- and Autocorrelation}
	\gra[0.7]{correlation}{Visual comparison of convolution, cross-correlation and autocorrelation  \footnotemark} \footnotetext{By Cmglee - Own work, CC BY-SA 3.0, https://commons.wikimedia.org/w/index.php?curid=20206883}
\end{frame}
\begin{frame}
	\frametitle{Fourier Interferometry}
	Blablabla
\end{frame}
\begin{frame}
	\frametitle{Fourier Interferometry - Experimental Set Up}
   \gra[0.8]{Versuchsaufbau_4}{Experimental set-up for the cross-correlation measurement  \footfullcite{Bamberger}}
\end{frame}
\begin{frame}
	\frametitle{Fourier Interferometry - Experimental Set Up}
	\gra[0.8]{aufbau4}{Experimental set-up for the cross-correlation measurement}
\end{frame}
\begin{frame}
		\frametitle{Fourier Interferometry - Experimental Procedure}
	\begin{itemize}
		\item Exposure: $8\,\mathrm{minutes}$
		\item Development: $3\,\mathrm{minutes}$
		\item Pre-watering: $10\,\mathrm{seconds}$
		\item Watering: $2\,\mathrm{minutes}$
		\item Bleaching: $45\,\mathrm{minutes}$
		\item Pre-watering: $3 \times 30\,\mathrm{seconds}$
		\item Watering: $1\,\mathrm{minutes}$
	\end{itemize}
\end{frame}
\begin{frame}
	\frametitle{Fourier Interferometry - Results}

\begin{figure}
	\centering
	\begin{overpic}[width=0.3\textwidth,tics=10]
		{../figures/fourier-0-2-edit.png}
		\put(10,85){\Large\textcolor{white}{$\alpha=0\degree$}}
	\end{overpic}
	\begin{overpic}[width=0.3\textwidth,tics=10]
		{../figures/fourier-15-2-edit.png}
		\put(10,85){\Large\textcolor{white}{$\alpha=15\degree$}}
	\end{overpic}\\
		\vspace{0.2 cm}
	\begin{overpic}[width=0.3\textwidth,tics=10]
		{../figures/fourier-30-2-edit.png}
		\put(10,85){\Large\textcolor{white}{$\alpha=30\degree$}}
	\end{overpic}
	\begin{overpic}[width=0.3\textwidth,tics=10]
		{../figures/fourier-45-2-edit.png}
		\put(10,85){\Large\textcolor{white}{$\alpha=45\degree$}}
	\end{overpic}
\end{figure}
\end{frame}
\begin{frame}
\frametitle{Fourier Interferometry - Results}
\begin{figure}
	
	\begin{overpic}[width=0.3\textwidth,tics=10]
		{../figures/fourier-60-2-edit.png}
		\put(10,85){\Large\textcolor{white}{$\alpha=60\degree$}}
	\end{overpic}
	\begin{overpic}[width=0.3\textwidth,tics=10]
		{../figures/fourier-75-2-edit.png}
		\put(10,85){\Large\textcolor{white}{$\alpha=75\degree$}}
	\end{overpic}\\
	
	\vspace{0.2 cm}
	
	\begin{overpic}[width=0.3\textwidth,tics=10]
		{../figures/fourier-90-2-edit.png}
		\put(10,85){\Large\textcolor{white}{$\alpha=90\degree$}}
	\end{overpic}

\end{figure}
\end{frame}
\begin{frame}
	\frametitle{Fourier Interferometry - Comparison with Theoretical Rsults}
\graTwoOne[0.4]{Korrelation_Spalt1}{Korrelation_Spalt2}{Theoretical Predictions}
\end{frame}

\section{Summary and Discussion}
\frame{\tableofcontents[currentsection]}
\begin{frame}
	\frametitle{Summary of our Results}
	Blablabla
\end{frame}
\end{document}