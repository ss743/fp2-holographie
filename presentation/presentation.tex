\documentclass{beamer}

\usepackage{ufcd}
\usepackage[english]{babel}
\usepackage[backend=biber,backref=false,style=numeric-comp, sorting=none,block=ragged,firstinits=true]{biblatex}

\addbibresource{fp_refs.bib}

\usetheme{ufcd}

\title{Holography}
\author{Saskia Bondza \& Simon Stephan}
\date{04.04.2017}

\begin{document}
\maketitle
\frame{\tableofcontents}
\section{Theoretical Background}
\frame{\tableofcontents[currentsection]}
\begin{frame}
	\frametitle{Holography - How does it work?}
	Blablabla\footfullcite{staats}
\end{frame}
\begin{frame}
	\frametitle{Interference}

\end{frame}
\begin{frame}
	\frametitle{Coherence}
\end{frame}
\begin{frame}
	\frametitle{Hologram of an object point}
\end{frame}
\begin{frame}
	\frametitle{Types of Holography}
\end{frame}
\begin{frame}
	\frametitle{Reconstruction of a Hologram}
\end{frame}
\begin{frame}
	\frametitle{Holographic Interferometry}
\end{frame}
\begin{frame}
	\frametitle{Apparatus}
\end{frame}
\section{Procedure and Evaluation}
\frame{\tableofcontents[currentsection]}
\subsection{Michelson Interferometer}
\frame{\tableofcontents[currentsubsection]}
\begin{frame}
	\frametitle{Michelson Interferometer}
	Blablabla
\end{frame}
\subsection{Double Exposure Hologram}
\frame{\tableofcontents[currentsubsection]}
\begin{frame}
	\frametitle{Double Exposure Hologram}
	Blablabla
\end{frame}
\begin{frame}
	\frametitle{Calculation of the Elastic Modulus}
	Blablabla
\end{frame}
\subsection{Real Time Hologram}
\frame{\tableofcontents[currentsubsection]}
\begin{frame}
	\frametitle{Real Time Hologram}
	Blablabla
\end{frame}
\begin{frame}
	\frametitle{Oscillations of an Aluminium Plate}
	Blablabla
\end{frame}
\subsection{Fourier Interferometry}
\frame{\tableofcontents[currentsubsection]}
\begin{frame}
	\frametitle{Fourier Interferometry}
	Blablabla
\end{frame}
\begin{frame}
	\frametitle{Fourier Optics - Huygen's Principle and Fraunhofer Diffraction}
	Blablabla
\end{frame}
\begin{frame}
	\frametitle{Convolution Theorem}
	\begin{align}
	(f*g)(t)=\int\limits_{-\infty}^{\infty} f(\tau)\cdot g(t-\tau) d\tau \label{Convo1}
	\end{align}
%	\begin{align}
%	\mathscr{F}(f*g)=const. \mathscr{F}(f)\cdot \mathscr{F}(g)
%	\end{align}
%	 \graX[0.8]{Convolution}{Illustration of the Convolution theorem}{Illustration of the Convolution theorem with the example of the double slit \label{Convo} \footnotemark}\footnotetext{http://www4.uwsp.edu/physastr/kmenning/images/Hecht4.11.F.31.png}
\end{frame}
\begin{frame}
	\frametitle{Cross- and Autocorrelation}
%	\begin{align}
%	(f*g)(t)=\int\limits_{-\infty}^{\infty} f(\tau)\cdot g(t+\tau) d\tau \label{corr}
%	\end{align}
%	\graX[0.5]{correlation}{Visual comparison of convolution, cross-correlation and autocorrelation}{Visual comparison of convolution, cross-correlation and autocorrelation \label{corsscor} \footnotemark} \footnotetext{By Cmglee - Own work, CC BY-SA 3.0, https://commons.wikimedia.org/w/index.php?curid=20206883}
\end{frame}
\begin{frame}
	\frametitle{Experimental Set-up}
	Blablabla
\end{frame}
\begin{frame}
	\frametitle{Experimental Procedure}
	Blablabla
\end{frame}
\section{Summary and Discussion}
\frame{\tableofcontents[currentsection]}
\begin{frame}
	\frametitle{Summary of our Results}
	Blablabla
\end{frame}
\end{document}